\chapter{Vulnerabilities}
\section{Nomenclature}
 - Bug: error or fault that causes a failure
 - Error: human action that produces an incorrect result
 - Fault: incorrect step or process in a computer program (defect or flaw that can lead to a failure)
 - Failure: inability of software to perform its functions within required performance constraings (visible cause of a failure)

See: \textit{Not all that glitters is gold}

\section{Vulnerabilities classes}
 - information leakage: unintented disclosure of secret information at the end user due to security breach
 - buffer overflow
 - race condition
 - invalid data processing

\section{Memory Corruption}
Covered memory corryption attacks:
\begin{itemize}
  \item buffer overflow
  \item heap overflow
  \item shellcode injection
\end{itemize}

\subsection{Typical Errors}
Out-of-bound access, buffer overflow, reference pointing to the wrong location.

Variables overriding consists in smashing the variables allocated on the stack alongside some buffer, under user control.
This attack is used to corrupt the return address of a function.

In 64 bits binaries, the entire 64 bits address space is not used in its entirety; instead, a 48 bits (the least significant 48 bits) address space is used.

Addresses must be in \textbf{canonical form}:
\begin{itemize}
  \item From 0x0000000000000000 to 0x00007FFFFFFFFFFF -> application code
  \item From 0xFFFF800000000000 to 0xFFFFFFFFFFFFFFFF -> kernel code (not copyied, virtual mem manages it)
\end{itemize}

\subsection{Heap Corruption}
The typical objective is mangling data structures. E.g., changing point of a linked list. 

\subsection{Code Injection}
Idea: inject code using a buffer overflow and jumping to that portion of the stack. \\
Requirements: ASLR + NX (stack areas are mutually executable or writable) deactivated.

No stack canary means build with \texttt{-fno-stack-protector}.

See: PwnCollege "Shellcode Injection" and "Intermediate memory errors" training modules

Typically shellcraft is used to generate the shellcode.

In order to locate the offset between the beginning of user input (input buffer) and return address we can use de Brujin sequences.
\texttt{cyclic <num>} provides the means to inject the sequence, while \texttt{cyclic -l 0x<seq-hex>} locates the distance from input buffer start, thanks to low repeatability of the sequence.

Pwntools can be used to perform binary attacks:
\begin{verbatim}
from pwn import *

exec = ELF("<binary-name>")

payload = "a"*32  # offset
payload += p32(<target-addr>)
payload += asm(shellcraft.sh())

proc = process("./<bin-name>")
proc.sendlineafter(b'<str>', payload)

print(proc.clean()) # Clean returns the output of the process accumulated up to this point
proc.interactive()  # Give control to the user (since we opened a shell)
\end{verbatim}

\subsection{ASLR Bypass}
ASLR randomizes addresses with a granularity of 4k (virtual memory size).

\section{Format String Attacks}
First documented in the second half of 2000. See: Team Teso (2001), Format String Exploitation Tutorial by Saif El-Sherei.

Printf, scanf behaviour can be manipulated by carefully crafting format strings.
Format strings can allow to read and write values from and to the stack.

\texttt{Xprintf} and similar functions are variadic functions that take as first arg a format string.

Eg: \texttt{printf(argv[1])} prints an arbitrary string. This string can even be a format string.
Trying to pass \texttt{\%s\%s\%s} as argument, a segmentation fault will occur.
Why? Since it is a variadic function, the addresses of the 3 strings will be searched into the stack.
Printf function expects to find the values to substitute the placeholders in its internal stack.

\texttt{\%x} is the placeholder intended to print a value that has the width of a machine word, in hexadecumal format.

Modifiers can be used to alterate the length of the word: \textt{h} is the modifier for short, while \texttt{l} is the one for long.
\texttt{\%<space-num>x} prints the same number of bytes as before but aligned to \texttt{<space-num>} bytes, while \texttt{\%<num>\$<space-num>x} jumps to the num-th byte on the stack (arbitrary stack data leak).

Keep an eye on stack alignment, which depends on the underlying architecture.

Idea: once the jump location has been identified (e.g. a return address), its content can be overwritten using the \texttt{\%n} operator.

The \texttt{\%n} placeholder emits an integer value that corresponds to the number of chars prior the placeholder.
E.g. in \textit{"Hello \%n"} the number 6 is emitted. For big jumps, this is more useful \textit{"\%<num>c \%n"}.
The location where \texttt{\%n} places the value is the last address found in the last element of the variadic stack.

The idea is to generate the address we want to write on in the stack, read up to that point, then perform a \texttt{\%n}.

See: GCC fortify \\
Remember to disable ASLR

In real-world attack, the address of PLT/GOT table is found, then the address of system functions is written (in two steps since the address is large) as a return address.
This allows arbitrary invokations.

See: meaning of FORTIFY_SOURCE=2

Safe version: \texttt{printf("\%s", user_input)}

\section{Countermeasures}
\subsection{Non-Executable Stack}
Code injection attacks can be mitigated with NX | Non eXecutable | stack. The stack is flagged as RW and not RWE (Read Write Execute).

For most compilers, the default is stack hardening. Though, GCC's \texttt{-z noexecstack} can be used. \texttt{readlef -e <binary>} can be used to verify with which countermeasures the binary is compiled.

NX stack is helpless if code is not executed from the stack. E.g., return to libc and ROP chains.

\subsection{Stack Canaries}
The stack is usually instrumented in such a way that a secret value, changing at each run, is written near the return address of the current stack pointer at the beginning of the function and check at the exit.

Without \texttt{-fstack-protector} the canary is inserted smartly: if a buffer overflow is possible in the stack frame it is inserted, not inserted otherwise.
With that option it is always inserted.

This mitigation can be broken in at most 7*256 tries if the server behaves like an oracle that tells us if a stack smashing happened.
The idea is that lower byte is always zero to stop reading of the canary with printf/puts functions; the remaining 7 bytes can be bruteforces if the server application spawnns a thread each time a new client arrive (since process image is the same for both parents and childs).

Leaking a canary is also possible with a format string attack, by dumping the content of some RAM.

\subsection{ASLR}
Address Space Layout Randomization randomly positions base address of text section, stack and heap spaces.

Remember that libraries are often shared through CoW mechanisms.
If ASLR wasn't on we could work out the offset of, for instance, \texttt{system} function and use it in a ROP chain.

Different ASLR levels can be chosen. See: memory maps in \texttt{/proc/<pid>/}

See: \texttt{randomize_va_space} levels (0 disabled; 1 randomizes stack, VDSO adn shared memory; 2 randomizes the prev plus data segment and heap).

\href{Personalities}{https://man7.org/linux/man-pages/man2/personality.2.html} are the way to granularly change system properties for a single process. The \texttt{setarch} command is used.

Bypassing ASLR:
\begin{itemize}
  \item bruteforce all possible addresses: feasable in x86 (4k of randomness) but not in x86_64
  \item exfiltrate the address of a libc function (or any other function). Depending on in which section the element is stored, I can break different parts of ASLR (text, data, etc.). *
\end{itemize}

* remember to use the version of the lib that exactly binds with the one used in the binary.

\subsection{Control-Flow Integrity}
A call graph (or control flow graph) is computed at compile-time. If the runtime control-flow graph differs, the binary is killed.

See: https://eli.thegreenplace.net

\subsection{PLT and GOT: Partial and Full RELRO}
Dynamic linking is performed with two support tables:
\begin{itemize}
  \item GOT -- Global Offset Table: huge table that contains in each the address of each global object.
  N elements mapped to N addressees.
  \item PLT -- Procedure Linkage Table: table that contains a trampoline for each dynamically linked function.
  First PLT object is the solver.
  PLT entries are trampolines that resolve the address of the function at runtime the first time, then store the target addr of the function in the GOT table (lazy).
\end{itemize}

If a runtime address is leaked, the address of GOT table can be worked out.
Each function that accepts an address as its first and only argument, is a potential candidate for the attacker.
The attacker can overwrite the address of such a function | for e.g. with heap overflow | and redirect the control flow to the desired target.

Two countermeasures are used by modern Linux OSs:
\begin{itemize}
  \item Partial RELRO: GOT is stored before any writable memory section (before the heap ), then it cannot be overwritten by a heap overlow
  \item Full RELRO: the GOT table is solved "now" (DT_BIND_NOW), at the start of the binary, and then set as read-only.
\end{itemize}

GCC options are: \texttt{-Wl,-z,norelro} to disable it, \texttt{-Wl,-z,relro} to enable partial RELRO, and \texttt{-Wl,-z,relro,-z,now} to enable full RELRO.

See: https://www.redhat.com/en/blog/hardening-elf-binaries-using-relocation-read-only-relro.

Relocation records can be listed with: \texttt{objdump -r <binary>}.

\section{Bypassing Countermeasures}
\subsection{Return to libc}
Objective: bypass NX using libc content to run a shell.

Remember: x32 ABI involves pushing function params in reverse order on the stack, then the return address, which is used by the ret to return control flow to the caller.

To find the address of \texttt{system} function call \texttt{p system} after the start of the program (after ld loaded dynamic libraries).

Libc always contains "/bin/sh" string, since it is used by \texttt{system}, which under the hood calls \texttt{execve} with "/bin/sh" as first argument.

\texttt{strings -t x <libc-so>} to print the offset of "/bin/sh" in hexadecimal.

\texttt{info proc mappings} prints the memory maps of the application.

Idea: inject into the stack padding + ret address (to libc) + first argument (addr of "/bin/sh").

Remember: \texttt{bash -i} to get an interactive shell.

Ret2libc allows for a chain of execution long at most 2: junk + system addr + exit address + shell string address + exit code

See: https://docs.pwntools.com/en/stable/elf/corefile.html

\subsection{ROP Oriented Programming}
Idea: push on the stack address of functions I want to execute or addresses of fragments of code that prepare/shift the stack for something else.

These fragments are called gadgets: blocks of instructions that at the end returns/branches.

See: \texttt{ROPgadget.py}. Usage: \texttt{ROPgadget.py --binary <binary/libc-so>}.

We are especially looking for gadgets that perform register-to-register, register-to-memory, memory-to-registers operations, since in x86_64 arguments are passed through registers.

Basically we are using stack and gadgets as a Turing tape.

In x86 gadgets are also useful to realign the stack and prepare params for the next call, see: pop-pop-pop-ret sequence.

See: Weird Machine

\subsection{GOT Overwrite}
This attacks can be performed on binaries with no or partial RELRO.
The objective is to overwrite the address of a libc function contained in GOT talbe (for eg, through fmt string attack) that accepts arguments (such as puts/printf), and substitute it with a function like \texttt{system}.

To get GOT addres address from GDB (32-bits binarie) we have to read \textit{<function>@plt} interpreting it as instructions.
The result is sequence of push-jmp. The address towards which the jump is performed is the address of the entry in GOT table.