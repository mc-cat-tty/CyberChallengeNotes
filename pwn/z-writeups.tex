\chapter{Pwn Writeups}

\section{Software Security 0}
\subsection{SS_0.01 - The safe}
Idea: password can be found in plaintext in \texttt{.rodata} section.

Run: \texttt{readelf -p .rodata the_safe}

\subsection{SS_0.02 - acrostic}
Literally: \texttt{objdump -d acrostic}

\subsection{SS_0.03 - dissection}
Idea: list all sections contained in the provided ELF file. The flag is encoded in sections name.

Run: \texttt{readelf -S dissection | grep CCI -A 17 | sed -n "p;n" | awk '{print $2}' | tr -d '\n'}

\subsection{SS_0.04 - volatility}
Idea: put a breakpoint at the beginning of one of the functions that receive the flag as argument, then print the first and only arg.

Run: \texttt{gdb --batch --command=scripts/volatility.gdb volatility}

\subsection{SS_0.05 - piecewise}
See: \textit{scripts/piecewise.py}

\section{Software Security 1}
\subsection{SS_1.01 - NextGen Safe}
Idea: the function that prints the flag is not called, however is packed into the binary (check with \texttt{readelf -p .strtab nextgen_safe}, or \texttt{info functions} inside GDB).
Sooo... just call it from GDB.

Run: \texttt{gdb --batch -x scripts/nextgen_safe.gdb nextgen_safe}

\subsection{SS_1.02 - Slow Printer}
Idea: open program in GDB and intercept every syscall to \texttt{clock_nanosleep}.
As soon as the system call is catched, set \texttt{RDX} register to \texttt{null}.
\texttt{RDX} register corresponds to \texttt{const struct timespec *t} parameter of \texttt{clock_nanosleep} function (ref. \href{https://man7.org/linux/man-pages/man2/clock_nanosleep.2.html}).

Run: \texttt{gdb --batch -x slow_printer.gdb slow_printer}

Alternatives:
\begin{itemize}
  \item skip \texttt{clock_nanosleep} calls patching the binary with NOPs (e.g., with Ghidra). See \textit{scripts/slow_printer_patched}
  \item DLL (\texttt{LD_PRELOAD}) injection
\end{itemize}