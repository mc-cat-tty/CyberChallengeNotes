\chapter{Pwn Writeups}

\section{Software Security 0}
\subsection{SS_0.01 - The safe}
Idea: password can be found in plaintext in \texttt{.rodata} section.

Run: \texttt{readelf -p .rodata the_safe}

\subsection{SS_0.02 - acrostic}
Literally: \texttt{objdump -d acrostic}

\subsection{SS_0.03 - dissection}
Idea: list all sections contained in the provided ELF file. The flag is encoded in sections name.

Run: \texttt{readelf -S dissection | grep CCI -A 17 | sed -n "p;n" | awk '{print $2}' | tr -d '\n'}

\subsection{SS_0.04 - volatility}
Idea: put a breakpoint at the beginning of one of the functions that receive the flag as argument, then print the first and only arg.

Run: \texttt{gdb --batch --command=scripts/volatility.gdb volatility}

\subsection{SS_0.05 - piecewise}
See: \textit{scripts/piecewise.py}

\section{Software Security 1}
\subsection{SS_1.01 - NextGen Safe}
Idea: the function that prints the flag is not called, however is packed into the binary (check with \texttt{readelf -p .strtab nextgen_safe}, or \texttt{info functions} inside GDB).
Sooo... just call it from GDB.

Run: \texttt{gdb --batch -x scripts/nextgen_safe.gdb nextgen_safe}

\subsection{SS_1.02 - Slow Printer}
Idea: open program in GDB and intercept every syscall to \texttt{clock_nanosleep}.
As soon as the system call is catched, set \texttt{RDX} register to \texttt{null}.
\texttt{RDX} register corresponds to \texttt{const struct timespec *t} parameter of \texttt{clock_nanosleep} function (ref. \href{https://man7.org/linux/man-pages/man2/clock_nanosleep.2.html}).

Run: \texttt{gdb --batch -x slow_printer.gdb slow_printer}

Alternatives:
\begin{itemize}
  \item skip \texttt{clock_nanosleep} calls patching the binary with NOPs (e.g., with Ghidra). See \textit{scripts/slow_printer_patched}
  \item DLL (\texttt{LD_PRELOAD}) injection
\end{itemize}

\subsection{SS_1.03 - Flag Checker}
Run code in GDB: \texttt{gdb flag_checker}.
Start it with a string as first arg: \texttt{start test}.
A SEGV will occur; backtracing it (\texttt{bt}) show an infinite recursive call.

Dissecting the binary (eg., in Ghidra) shows a function that computes the cumulative sum (stored in its third arg) of the first argument passed to the function (argv[1]).
Each character of the cumsum is tested against what presumably is each character of the flag's cumsum.
By reading the memory location where this array of integers is stored and reversing the cumsum the flag is disclosed.

See: \textit{scripts/flag_checker.py}

\subsection{SS_1.04 - Unbreakable AES}
By disassembling the code three internal functions are found \texttt{main}, \texttt{encrypt} and \texttt{ror}.

\texttt{main} function simply performs file handling operations (basically opening), as well as key validity checking (only alphanum chars).

If successful, \texttt{encrypt(<input-file>, <output-file>, <key>)} is invoked.
By carefully analyzing the disassembled code of this function, it turns out that key is not used.
The function performs right rotation of each character of the ciphertext (input file) a number of times equal to the character position plus one.

See: \textit{scripts/unbreakable_aes.py} is the script used to implement the decryption, making use of the inverse function of ror: rol -- ROtate Left.

\subsection{SS_1.06 - pacman}
Static analysis reveals that the binary implements a game in which the player has to navigate a maze with \texttt{hjkl} controls.
The final goal of the game is traversing the maze without touching the letters that compose the word "pacman": 'p', 'a', 'c', etc.

As an alternative, the binary can be dynamically analyzed; however, a kinda protection is enabled at the beginning of the \texttt{main} function: \texttt{ptrace(request: PTRACE_TRACEME, 0, 1, 0) == -1}.
This protection can be easily bypassed patching the binary. This can turn in useful if you are willing to profile its execution under GDB, Ptrace, Strace, and so on.

A suitable sequence of moves can be found implementing a simple backtracking algorithm.

See: \textit{scripts/pacman.py}

\section{Software Security 2}
\subsection{SS_2.01 - Digital billboard}
The following snippet is the core of the challenge:
\begin{verbatim}
struct billboard {
    char text[256];
    char devmode;
};
struct billboard bb = { .text="Placeholder", .devmode=0 };

void set_text(int argc, char* argv[]) {
    strcpy(bb.text, argv[1]);
    printf("Successfully set text to: %s\n", bb.text);
    return;
}
\end{verbatim}

Using \texttt{strcpy} is inherently unsafe.
By maliciously manipulating the input string, \texttt{devmode} flag can be flipped.
The value of \texttt{text} is comnpletely under the user's input: \texttt{'a'*257} is enough to overflow the allocated buffer, thus turning \texttt{devmode} into \texttt{'a'}, which is 1 under boolean perspective.

\subsection{SS_2.02 - 1996}
Objective: change control flow to run the \texttt{spawn_shell} function.

\texttt{cin} function allows stack smashing, and PIE is disabled, making it impossible for the kernel to enable ASLR.
\texttt{spawn_shell} address can be found from the symbol table: \texttt{objdump -t 1996 | grep "spawn"}.
The address is 64-bits long, like \texttt{file 1996} confirms.

Reversing the binary allows us to understand the offset of the stored RIP in \texttt{main}'s stack: 1048 bytes from buffer's base address.
By filling the stack with 1048 bytes of padding and appeding the desired address (\textit{0000000000400897}) at the end of the payload, the challenge is solved.

See: \textit{scripts/1996.py}

\subsection{SS_2.03 - the answer}
Objective: change the value of \texttt{the_answer} variable in \texttt{.data} section, replacing its content with the value 42.

From the application source code, call to \texttt{printf} with arbitrary user input can be noticed.
The user input is fed to the function as a first argument, thus enabling format strings interpretation.
A format string attack is therefore chosen.

Address of the variable is extracted from the symbol table: \texttt{objdump -t the_answer | grep answer}. \\
Result: \textit{0x0000000000601078}

Solution#1: \texttt{python3 -c 'import sys; from pwn import p32; sys.stdout.buffer.write(b"\%42c\%12\$ln".ljust(16, b"A") + p32(0x0000000000601078).ljust(8, b"\0") + b"\n")' | ./the_answer}
Solution#2: \texttt{python3 -c 'import sys; from pwn import p32; sys.stdout.buffer.write(b"AAAAAA\%36c\%12\$ln" + p32(0x0000000000601078).ljust(8, b"\0") + b"\n")'}

The attack consists in writing 42 bytes, then jumping to printf variadic stack that contains the target address, writing emitted bytes/chars (42) in that location with \texttt{\%12\$ln}.

\subsection{SS_2.04 - restricted shell}
Objective: get an unrestricted shell on the target machine.

Preliminary analysis: it can be noticed (with \texttt{checksec}) that the binary has all security measures turned off, especially the NX flag. \\
Performing a static analysis on the binary reveals that \texttt{shell()} functions allocate a buffer of 40 bytes and write user's input in it, with an unbounded function (\texttt{gets}). \\
The application terminates its execution if the passed command does not match \texttt{ls} or \texttt{dir}; if that's the case, then the function returns with an error message. \\

Given the unsecure nature of \texttt{gets} a buffer overflow can be performed, with the intermediate objective of overwriting the return address of the function. \\
The static analysis also reveals that a "gadget" | \texttt{jmp esp} | is placed at the end of the function. This instruction (combined with the fact that NX is disabled) allows us to move control flow to the stack. \\

This scenario leads to an attack that exploits the buffer overflow to redirect execution towards the gadget, which in turn executes user-controlled input on the stack, where a shellcode will be placed.

It is also worth noticing that before the \texttt{ret} instruction, a \texttt{leave} is present, which changes the esp-ebp "configuration" in the midst of the attack.

The final pyaload is crafted as follows:
\begin{itemize}
  \item 44 bytes of padding -- some NOPs has been used, but even non-valid opcodes are ok. This offset can be found either by hand (static analysis) or with \texttt{pwn cyclic} (dynamic analysis).
  \item the target address -- aka the address of the gadget.
  \item the shellcode (which, after the leave+ret, will be found at the head of the stack)
\end{itemize}

See: \textit{scripts/restricted_shell.py}

\subsection{SS_2.05 - LMRTFY}
The input buffer is executable and gets executed in \texttt{main} function.
System calls are blacklisted by \texttt{memmem} functions which search 3 needles (\texttt{int 80}, \texttt{syscall}, \texttt{sysenter}) in input buffer.

Idea: use a standard shellcode to open a shell, replacing \texttt{int 80} with an absolute jump (actually, a push and ret) to an instruction that performs it, a "gadget".

See: \textit{scripts/lmrtfy.py}

\subsection{SS_2.06 - arraymaster1}
Static analysis: the application was probably written in C.
For each array the program keeps track of 5 piece of data of length 8 bytes: len (+0), type (+8), p* (+16), get (+24), set (+32).
This control structure contains 3 attributes (len, type, pointer to the data region -- called p*) and 2 methods (get and set).
Debug symbols are not stripped; by looking at them, \texttt{spawn_shell} function can be found.
The input length of an array is unbounded, everything else looks safe: puts used instead of printfs with no format string set and get 
However, it can be noticed that data length is stored in the struct directly as the user input; the malloc size is instead \texttt{len * data_bits/8}. 
The critical point of the application is the \texttt{imul} before the call to the \texttt{malloc}.

Dynamic analysis: dynamically tracing calls to \texttt{malloc} with \texttt{ltrace} reveals that the heap is 8-bytes aligned and, as far as slab control is concerned, each data is preceded by the a block of 8 bytes.

Idea: exploiting the fact that len is stored as-is in the struct, while before the malloc a multiplication is performed, allowing an integer overflow to happen.
In particular, our objective is to store a big size in the control struct, to unlock arbitrary writes in the heap with the set function, while at the same time allocating a small amount of bytes.
Such length values depend on the size of the data type (64 bits in the script). \\
Once the length check in set function is bypassed, we can basically write arbitrary bytes on arbitrary heap (maybe even stack, not experimented) locations allocated after our "array".
By allocating a second "array" and overwriting its \texttt{set} function pointer with \texttt{spawn_shell} address, we will get a shell right after the invocation of the set function of the second block.


See: \textit{scripts/arraymaster.py}

\section{Software Security 3}
\subsection{SS_3.01 - ReallyOptimizedPrimality test}
Idea#1: since the NX flag is enabled, injecting a shellcode would be useless.
As a first approach, ret2libc has been used; however, it turned out that wasn't the proper solution since ASLR is allegedly active on the remote machine.

Idea#2: since ret2libc didn't work out, a ROP chain has been used.
"/bin/sh" string can be found in the \texttt{.rodata} section of the binary, as well as a "pop eax; int 80" sequence; plus, a series of other gadgets are available to set edx and ecx registers.
The idea in therefore to build a ROP chain that sets registers in such a way that the "int 80" successfully calls an \texttt{execve} with arguments \texttt{("/bin/sh", NULL, NULL)}.

The following diagram illustrates the payload:
\begin{verbatim}
+----------+----------------------------+---------------+--------+--------------------+--------+-----------------------+--------+
|  'A'*80  |  POP_EBX_POP_ECX_RET_ADDR  |  BIN_SH_ADDR  |  0x00  |  POP_EDX_RET_ADDR  |  0x00  |  POP_EAX_INT_80_ADDR  |  0x0b  |
+----------+----------------------------+---------------+--------+--------------------+--------+-----------------------+--------+
\end{verbatim}

See: \textit{scripts/primality.py}

\subsection{SS_3.02 - Eliza}
By statically analyzing the code it can be noticed that the binary was compiled with hardened canaries (even where buffers are not allocated).
Another important thing to be aware of is that the input-check-output sequence is contained in an infinite superloop.

Idea: leaking the canary overwriting its most significant byte (0 to stop printing functions).
After leaking the canary we can craft a payload that smashes the stack, passing canary checks and leading to \texttt{sp4wn_4_sh311} execution.

\begin{verbatim}
buf (rsi): 0x7fffffffe100
ret (rsp): 0x7fffffffe158
diff: 88

Stack layout:
+-----|--------+---------+---------+
| buf | canary | old_rbp | old_rip |
+-----|--------+---------+---------+
0     +72      +80       +88

sp4wn_4_sh311: 0x400897 (fixed since PIE is off).
\end{verbatim}

See: \textit{scripts/eliza.py}

\subsection{SS_3.03 - TicTacToe}
We have to overwrite a function that has the same signature as the one that we want to overwrite.
We will overwrite the puts that prints the name of the gamer.

Changing the content of the \texttt{puts}'s address in the GOT table \textbf{after a resolution} (this is fundamental to avoid the right resolution to overwrite our malicious value) will result in the next execution of \texttt{system} instead of \texttt{puts}. 

Notice: \texttt{ltrace} works by hooking to the linked library function in the PLT/GOT table.

Two \texttt{system}s are available: \texttt{system@plt} and \texttt{.got.plt.system}. The same holds for \texttt{puts}.

Both \texttt{system@plt} and \texttt{puts@plt} are known at compile time.

The format string is crafted as follows: \texttt{<got.plt["puts"]>\%<num-char-equivalent-to-system@plt-addr>c<padding-x*4-bytes>\%15\$ln}

See: \textit{scripts/tictactoe.py}

\subsection{SS_3.04 - Try-your-luck}
Static analysis of the binary reveals that the buffer allocated for the user's name is undersize, allowing a buffer overflow to occur when \texttt{read} function is called;
it is also evident that the final goal of the challenge is to return to \texttt{you_won} function in order to get a shell.

The binary has no canaries, but ASLR, PIE and NX are enabled.

Also, dynamic analysis confirms that the address of the function is randomized with a granularity of 4Kb (lower 12 bits).
At the same time, the higher bytes are the same as the one of the \texttt{game}'s function caller, aka \texttt{main} function.
In fact the whole program is so short that it lives in the same memory page.

Idea: it is sufficient to overwrite the lower 12 bits to invoke the \texttt{you_won} function.
This is however unfeasable, since we can write memory with a granularity of one byte: the smallest data size we can send to the program is then 16 bits.
Meaning, the fourth from last nibble (4 bits) will have to be bruteforced.

See: \textit{scripts/luck.py}