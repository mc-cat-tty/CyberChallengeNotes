\chapter{SRE/Debugging Tools}

\section{Debugging}
Debugging is the process of finding and (potentially) fix an executable by a series of means: dynamic analysis, profiling (syscalls, open network connections), and static analysis.

Given a binary file we can:
 - check if the file is executable
 - check target architecture (\texttt{file}/\texttt{readelf})
 - collect symbols (\texttt{readelf}) and strings (\texttt{strings})
 - check if there is a running process associated with the program
 - compare the SHA of the binary with known samples
 - identify function names and libraries

\texttt{strings} is tunable regarding the min and max length of the strings, the section to inspection, etc.

\texttt{readelf} can be used to easily read the header file of a ELF binary. 

\section{GDB}
See: GDB scripting

Editing the environment:
\begin{verbatim}
path <dir>
show paths
show environment [<var>]
set environment <var> <value>
unset environment <var>
\end{verbatim}

It sometimes useful to manipulate the environment passed to a binary.

Setting breakpoints:
\begin{verbatim}
  break <location>
  break  # next instruction
  break <location> [if <condition>]
\end{verbatim}

Breakpoints are usually implemented as interrupts in x86 architecture: a dedicated instruction is used to replace the instruction where we place the breakpoint.

A \textit{watchpoint} is the equivalent of a breakpoint... but for memory: the are triggered when a specific memory area is accessed in some mode (R or W).

A \textit{catchpoint} allows the debugger to stop the program when a certain event (exception, binary loading, etc.) occurs.
Especially useful to intercept system calls.

Hint: if a binary is compiled with PIE option, opt for displacements instead of addresses (they will change depending on the execution due to ASLR).

Displaying stack frames:
\begin{verbatim}
  frame  # Print current function frame content
  info frame  # Prints current frame information
\end{verbatim}

\texttt{info registers}, on the other hand, prints registers status. \texttt{p <reg>} prints a specific register

See: GEF, PwnGDB

\texttt{checksec <filename>} or, alternatively, pwntools version: \texttt{pwn checksec <filename>}

\texttt{x/<format>} (short for \texttt{display}) prints the data on the stack with the format specified in the \texttt{<format>} field.

Tipically format is \texttt{1gx}. 1 value, g for giant, x for hex. \texttt{s} can be used to print strings; \texttt{b} to print bytes.

\texttt{info functions} prints functions listed in the symbol table.

\section{Ghidra}
\textit{Thunk function} is a function whose only purpose is to pass control to another function.
