\chapter{Symmetric Crypto}

Misused protocols are often the key!

\section{DES Weaknesses}
Weak keys: keys such that D=E. They are often blacklisted (?)

Semi-weak keys: keys such that D(k1, E(k2, m))

2DES has been theoretically proposed but broken since just $2^{57}$ bruteforce attempts are required to find the right key.
Not $2^{112}$ as expected.

3DES is mainly used in 2 different modes: DDD and DED, depending if three or two keys are used.
In the latter case a key is recycled for both the D and E block.

AES has been selected during an american competition. The original name of the algo was Rijndael.

AES is usually attacked non in the inner part (cipher) but in the boundary configuration (protocol).

\section{Stream Ciphers}
Objective: use a cipher to encrypt a stream of data of arbitrary length. Mode of operations solve this.

\subsection{Electronic Codebook Mode (ECB)}
A naive idea is to split the data stream into 128-bit blocks. Each block encrypted separetely.
This is vulnerable to statistical analysis. Since the same plaintext block will give the same ciphertext block.
See: ECB Linux Penguin

Encryption Oracle: service that, given a plaintext, returns the corresponding ciphertext using the same key.
We can use ECB ciphers as long as we know that the input text is less or equal the size of a single block.

\subsection{Padding}
Padding: last block is often not fully filled. Fill the ramaining space with some data to align it to the desired length (e.g., 16 bytes).

Problem: cannot disambiguate between padding and plaintext at decipher time.

Idea: PKCS7. Fill remaining space with the length of the padding; last value will represent the padding length.
Inconvenience if the payload is aligned to the block size, since a new block will be allocated with the sole purpose of carrying padding length.

\subsection{Cipher Block Chaining (CBC) Mode}
Idea: make each ciphertext block dependent on the previous one, to scramble more data and solve the problem of same plaintext blocks encrypted in the same way.

Initialization vector (IV) is used (and should be changed each time) to make less predibile first block's encryption.
Sometimes a nonce, aka a sequential value, is used.

Using the key as both the key and the IV is not a good idea, it is a weak initialization.

\subsubsection{Padding Oracle}
Problem in odler Java versions on AES CBC + PKCS7: padding oracle.
Wether the padding was correct or not was disclosed (leaked) by the server.
This allows full disclosure of the full plaintext: starting from the first IV byte, we bruteforce it with the objective of making it result 1 in the plaintext.
If we find it, the padding oracle doesn't raise any exception. We then work out the value spit out from the encryption block.
We can now mould the first byte to be 2 and guess the second one, the oracle will tell us when it is correct.
Continuing... all the plaintext bytes will be disclosed.

\subsubsection{Bit Flipping}
In CRT modes the IV is public, aka is under control of the attacker.
This can be exploited during the decryption phase to perform a bit flip attack.
Bit flipping the ciphertext is also possible, but due confusion-diffusion principles it is harder to finely flip a single bit without creating garbage blocks.
Meaning, the desired bit can be flipped, but with some side effects.

On the contrary, a bit flip in the IV leads to a finely targeted bit flip in the plaintext.
Note that this is possible just for the first plaintext block.
Also note that arbitrary bit flips mean that, given that the original plaintext is known, it can be manipulated arbitrarily to turn it a certain value; e.g., turn the string "lollo" into "admin".

Remember: the XOR is "moudable"

Mitigation: sign the message or send it along with the hash of the message to prevent tampering

\subsection{Counter (CTR) Mode}
Streaming cipher. Extremely mouldable.

Idea: this time the core idea is to generate a kinda key of the same length of the plaintext (see perfect security definition), but without exchanging a key of such length.
Tries to imitate the OTP cipher.

How? it generates random data passing a nonce + counter on an AES block cipher. The output of AES blocks constitues the OTP "key".
This output is then xored with the plaintext to obtain the ciphertext.

Notice also that the output is cut at the length of the original text. This can disclose information about the content of the plaintext: see yes vs no length :)

If the nonce is kept constant, the encryption function will act as decryption for a given ciphertext:
\begin{verbatim}
c = block_out ^ p
c ^ block_out = p -> will give original plaintext, since block_out is constant
\end{verbatim}

\subsection{Galois Counter Mode}
AES CTR + signature

\subsection{ChaCha20}
Variant of Salsa20 published in 2008. Like AES CTR. Extremely mouldable.
