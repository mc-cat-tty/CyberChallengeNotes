\section{Introduction}
Reason: convenience (portability et sim. for DevOps) and security, to avoid that a pwned application exposes access to underlying host OS.

Containers are isolated environments sharing the same real operating system.
Containers are self-contained, meaning that they carry all the libraries/dependencies needed by the application to run.

Image vs container: image is immutable and static, while container is a running image. Same different between binary and process.

\subsection{Containers Lifecycle}
Images can be compiled on-premise or downloaded pre-compiled from a central repository/hub.
Same as building a binary from source code or instaling a packaged version.

Docker container states are running, paused, #TODO

\subsection{Commands}
run -> incapsulates pull, create, exec (from the image)
exec -> runs a command on a running container
ps -a -> all mean also stopped containers

Containers don't have storage, persistency is enabled by volumes.

\subsection{Persistency}
The option is the same (`-v`). If a path (either relative or absolute) is passed, a bind mount is performed. Otherwise 


\subsection{Dockerfile}
`FROM scratch`

Almost every directive is a different layer, for caching reasons.
Sometimes it is convenient to discard caching capabilities in favor of ligher images.

\subsection{Network}
Different configurations supported, main ones are:
 - bridge -> creates a isolated L3 network layer that allows containers to communicated with each other. Masquerading is 
 - host -> exposes ports to host network, host and containers share network properties (#TODO rephrase)

Explicit network none to avoid automatic creation of network, which allows containers to communicate.

Be aware that without specifying network address, 0.0.0.0 is used as default, making a service exposed to all the users in the LAN (or even all Internet if no NATting is hiding u).

\subsection{Compose}
Automates container creation from a YAML file.

Subcommands:
 - up -> runs services, even if not already started
 - start -> runs services, must be already running. U can add `-d` to detach the terminal
 - stop -> pauses the container
 - down -> pauses and deletes the container. U can add `-v` to delete the volumes

Build context provides variables to the Dockerfile that control conditionally the compilation, for instance.

\section{NS 0.00}
-i -> interactive
-t -> tty like termina

\section{NS 0.09}
`ss -tln' -> socket listening on tcp, number version of ports
`ss -tulpn'
`ss -puttane'